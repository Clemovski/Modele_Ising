\section*{Introduction}
\addcontentsline{toc}{section}{Introduction}

Le modèle d'Ising a été proposé en 1925 dans la thèse de Ernst ISING sous la direction de Wilhelm LENZ pour expliquer l'évolution de la chaleur spécifique des matériaux à basse température.\\
Il consiste en une assemblée de moment magnétiques de valeur $\pm$M (vers le haut ou vers le bas). Ces moments sont influençables par une excitation magnétique ou par leurs proches voisins.\\
On peut modéliser cette assemblée à une, deux ou trois dimensions pour s'approcher des conditions réelles, au prix de complexité et de temps de calcul.\\
\\
Une fois résolu, ce modèle permet de retrouver les phénomènes physiques suivants :\\
\begin{itemize}
\item Évolution de la chaleur spécifique du milieu en fonction de la température ou du champ magnétique.;
\item Évolution de la susceptibilité magnétique en fonction de la température ou du champ magnétique.;
\item Mise en évidence de la transition de phase.;
\item Évolution de la température critique en fonction de la taille du réseau ou du champ magnétique.;
\item Mise en évidence des domaines de Weiss.;
\item Phénomène d'hystérésis magnétique.;
\end{itemize}
\vspace{\parskip} % espace entre paragraphes

L'intérêt de la simulation informatique est que le modèle ne possède pas de solution mathématique exacte à 3 dimensions. Elle permet donc de voir de façon plus visuelle les effets du modèle et de l'influence des différents paramètres qui entrent en jeu.\\
Dans ce rapport nous allons voir comment on a pu modéliser informatiquement ce modèle et s'il est possible de s'en servir pour retrouver les phénomènes physiques que le Modèle d'Ising permet d'expliquer.\vspace{\parskip}

Mettre un plan ici ? ...