\section{Modèle Physique et simulation}


\subsection{Fonctionnement du modèle d'Ising}

Comme expliqué en introduction, le modèle d'Ising consiste en un assemblage d'éléments possédant chacun un moment magnétique positif ou négatif. ($\pm 1$ dans notre cas)\\

Dans le cas physique qui nous intéresse cet ensemble d'élément représente un système thermodynamique : un solide cubique dans le cas à 3 dimensions.\\
Les moments magnétiques sont ceux des électrons de valence qui composent le solide.\\
\\
Ces moments magnétiques sont influençables par leur voisins (nous nous limiterons aux premiers voisin dans cette étude) et par le champ magnétique auquel est soumis le système.\\
\\
Pour négliger les effets de bord, on utilise des conditions aux limites périodiques. C'est à dire que les éléments en bout de ligne agissent comme s'il étaient voisins des éléments en début de ligne.\\
\\
%\begin{equation}
%H = -J \sum_{i=1}^{N_{spins}} \sum_{<j>}^{} S_{i}S_{j} - \mu_{B} \sum_{i=1}^{N_{spins}} Si
%\end{equation}




\subsubsection{Observations attendues}
\item Évolution de la chaleur spécifique :\\


\item Évolution de la susceptibilité magnétique en fonction de la température ou du champ magnétique.;
\item Mise en évidence de la transition de phase.;
\item Évolution de la température critique en fonction de la taille du réseau ou du champ magnétique.;


\textbf{Mise en évidence des domaines de Weiss :}\\
Dans un matériau ferromagnétique, la constante de couplage J est positive et incite les spins à s'aligner avec leurs voisins.\\
En dessous de la température critique, les éléments ont alors tendance à former des zones de même spin. Ces zones sont appelées Domaines de Weiss du nom de Pierre-Ernest Weiss qui a prédit ce phénomène.\\


\textbf{Phénomène d'hystérésis magnétique :}\\
Lorsqu'on applique un champ magnétique au système, on incite les spins à suivre ce champ et ainsi à modifier le moment magnétique du matériau. C'est ainsi que le matériau est aimanté.\\
Cette évolution du moment induit est linéaire.\\
En appliquant un fort champ magnétique, on peut imposer un sens d'orientation à la quasi-totalité des spins. On arrive alors à une saturation où le moment du système n'évoluera plus.\\
\\
On peut penser que ce phénomène est réversible mais ce n'est pas le cas : Le couplage oblige les spins à garder l'orientation de leurs voisins et cette saturation n'est pas brisée avant d'atteindre un champ suffisamment fort dans l'autre sens. Le même phénomène se produit alors "en négatif".\\
\\
Le champ nécessaire pour ramener le solide à une aimantation nulle est appelé "coercitif" et l'aimantation que le matériau conserve lorsque l'excitation magnétique est revenue à zéro est dite "rémanente".\\

\vspace{\parskip} % espace entre paragraphes

\subsection{Simulation informatique}

\subsubsection{La méthode Métropolis}
Dans cette méthode on va faire évoluer le système à chaque instant, et ce indépendamment de ses états antérieurs.\\
Pour ce faire on va piocher un spin au hasard dans le système et envisager de le retourner en tenant compte des conditions présentes :\\
On définit une probabilité pour le spin de se retourner et on la compare à une valeur aléatoire.\\
Cette probabilité est donnée par la loi de Maxwell Boltzmann :\\
\begin{equation}
e^{- \dfrac{\Delta Ê}{k_B T}}
\end{equation}
où $\Delta E$ représente la différence d'énergie impliquée par le retournement du spin.\\
Cela correspond physiquement à une "bataille" entre l'énergie impliquée par les effets de spins (dans l'équation du Hamiltonien) avec l'agitation thermique. La probabilité de retournement est en faveur du changement d'énergie le plus petit.\\


\subsubsection{Structure et fonctionnement du programme}


Le programme possède 3 classes de solides : 1D, 2D et 3D. Ces trois classes dérivent d'une classe mère Solide qui contient les méthodes et attributs qui sont communes à toutes les dimensions.\\
\\
Le programme utilise une \emph{Factory} (SolideFactory) qui lui permet d'instancier un solide de n'importe-quelle dimension à l’exécution et d'ensuite pourvoir le manipuler sans avoir à se soucier de connaître cette dimension.\\
Cette \emph{Factory} est un \emph{Singleton} car une seule est nécessaire, et en avoir plusieurs pourrait poser plus d'ennuis qu'autre-chose.\\
\\
L'Imprimante enfin est l'interface avec l'utilisateur. C'est un \emph{Singleton} également et elle remplit les fonctions suivantes :\\
\begin{itemize}
\item Lecture des paramètres du solide à instancier et les conditions dans lequel le faire évoluer.
\item Ecriture des données récoltées lors de la simulation dans les fichiers mesures.txt et Weiss.txt
\item Génération des script à utiliser dans Gnuplot pour visualiser les données.
\end{itemize}
\vspace{\parskip} % espace entre paragraphes

Précisions :\\
La classe Solide possède des attributs redondants tels que champB et muBchampB ou temperature et kbT. Ces redondances servent à gagner un peu de temps de calcul en ne recalculant pas le produit de ces deux valeurs à chaque itération.\\
Les valeurs de la constante de Boltzmann et du magnéton de Bohr sont codées en dur car on part du principe qu'elle ne sont pas sujette à changer du jour au len-demain.\\
Toutes les grandeurs utilisées dans le programme sont homogénéisées de telle sorte que le Hamiltonien soit exprimé en électron-Volts.

