\section{Modèle Physique et simulation}


\subsection{Fonctionnement du modèle d'Ising}

Comme expliqué en introduction, le modèle d'Ising consiste en un assemblage d'éléments possédant chacun un moment magnétique positif ou négatif. ($\pm 1$ dans notre cas)\\

Dans le cas physique qui nous intéresse cet ensemble d'élément représente un système thermodynamique : un solide cubique dans le cas à 3 dimensions.\\
Les moments magnétiques sont ceux des électrons de valence qui composent le solide.\\
\\
Ces moments magnétiques sont influençables par leur voisins (nous nous limiterons aux premiers voisin dans cette étude) et par le champ magnétique auquel ils sont soumis.\\
\\
Pour négliger les effets de bord, on utilise des conditions aux limites périodiques. C'est à dire que les éléments en bout de ligne agissent comme s'il étaient voisins des éléments en début de ligne.\\





On peut mettre des mots en \emph{italique}, 

\subsubsection{Observations attendues}
\item Évolution de la chaleur spécifique du milieu en fonction de la température ou du champ magnétique.;
\item Évolution de la susceptibilité magnétique en fonction de la température ou du champ magnétique.;
\item Mise en évidence de la transition de phase.;
\item Évolution de la température critique en fonction de la taille du réseau ou du champ magnétique.;
\item Mise en évidence des domaines de Weiss.;
\item Phénomène d'hystérésis magnétique.;


\subsection{Simulation informatique}

\subsubsection{La méthode Métropolis}

\subsubsection{Structure et fonctionnement du programme}


Le programme possède 3 classes de solides : 1D, 2D et 3D. Ces trois classes dérivent d'une classe mère Solide qui contient les méthodes et attributs qui sont communes à toutes les dimensions.\\
\\
Le programme utilise une \emph{Factory} (SolideFactory) qui lui permet d'instancier un solide de n'importe-quelle dimension à l’exécution et d'ensuite pourvoir le manipuler sans avoir à se soucier de connaître cette dimension.\\
Cette \emph{Factory} est un \emph{Singleton} car une seule est nécessaire, et en avoir plusieurs pourrait poser plus d'ennuis qu'autre-chose.\\
\\
L'Imprimante enfin est l'interface avec l'utilisateur. C'est un \emph{Singleton} également et elle remplit les fonctions suivantes :\\
\begin{itemize}
\item Lecture des paramètres du solide à instancier et les conditions dans lequel le faire évoluer.;
\item Ecriture des données récoltées lors de la simulation dans les fichiers mesures.txt et Weiss.txt;
\item Génération des script à utiliser dans Gnuplot pour visualiser les données.;
\end{itemize}
\vspace{\parskip} % espace entre paragraphes

Précisions :\\
La classe Solide possède des attributs redondants tels que champB et muBchampB ou temperature et kbT. Ces redondances servent à gagner un peu de temps de calcul en ne recalculant pas le produit de ces deux valeurs à chaque itération.\\
Les valeurs de la constante de Boltzmann et du magnéton de Bohr sont codées en dur car on part du principe qu'elle ne sont pas sujette à changer du jour au len-demain.\\
Toutes les grandeurs utilisées dans le programme sont homogénéisées de telle sorte que le Hamiltonien soit exprimé en électron-Volts.

