\section{Résultats des simulations}
\subsection{Cas du matériau Ferromagnétique}
Le matériau Ferromagnétique a une constante de couplage J positive : Ses spins ont tendance à suivre l'orientation de leurs voisins.\\
C'est le cas du fer par exemple.\\

Évolution de la chaleur spécifique :\\


%\item Évolution de la susceptibilité magnétique en fonction de la température ou du champ magnétique.;
%\item Mise en évidence de la transition de phase.;
%\item Évolution de la température critique en fonction de la taille du réseau ou du champ magnétique.;


\textbf{Mise en évidence des domaines de Weiss :}\\
Les domaines de Weiss peuvent être visualisés en laissant le système évoluer spontanément à une température inférieure à la température critique. Ici, seuls les spins Up sont représentés en rose. Les spins Down sont devinés en négatif : Ils correspondent à la partie blanche.
\begin{center}
\includemovie{}{}{images/ani.gif}\\
\end{center}
On peut voir ici l'effet de la nature circulaire du solide : Les domaines se complètent si l'on met côte à côte cette image plusieurs fois.

\textbf{Phénomène d'hystérésis magnétique :}\\


\subsection{Cas du matériau Ferromagnétique}